% calldefs.tex
% documentation for SDR's call definition format.

\documentclass[12pt]{article}
\usepackage{tex-squares}%\setlength{\parindent}{1cm}
\usepackage{parskip}
\usepackage{url}
\usepackage{listings}
\title{The SDR call definition language}
\author{C. Scott Ananian}
\date{}

% define our own language for the listings package
\lstdefinelanguage{CallFile}
  {morekeywords=[1]{def:,in:,from:,call:,optional:,prim:,xpart:,part:,ipart:,spoken:,condition:},%
   morekeywords=[2]{LEFT,ANY,FACING,DANCERS,RH,MINIWAVE},%
   sensitive=true,%
   morecomment=[l]//,%
   morecomment=[s]{/*}{*/},%
   alsoletter={:},%
   alsodigit={/},%
  }[keywords,comments]%
\lstset{language=CallFile,columns=flexible,%
   identifierstyle=\ttfamily,keywordstyle=[2]\sffamily,%
   xleftmargin=20pt,xrightmargin=10pt,%
   breaklines=true,%
   breakatwhitespace=true,prebreak=\textbackslash,%
   escapechar=\#,%
   captionpos=b,abovecaptionskip=12pt
}

\newcommand{\clause}[1]{\texttt{#1}}
\renewcommand{\call}[1]{\texttt{#1}} % defined in the tex-squares package
\newcommand{\package}[1]{\url{#1}}
\begin{document}
\maketitle

\section{Introductory examples}

Although a few primitives and concepts are defined procedurally in the
SDR code base (see the \package{net.cscott.sdr.calls.lists}
package), the majority of the call and concept definitions in SDR are
defined using a simple human-readable language in resources stored
in \texttt{*.call} resource files in \package{net.cscott.sdr.calls.lists}.

\subsection{Calls defined in terms of other calls}
Here is a simple example call definition:
\begin{lstlisting}
def: dosado to a wave
  in: 6
    call: dosado
    call: step to a wave
\end{lstlisting}

This defines a call named ``dosado to a wave'' which completes in 6
beats of music (according to the standard Callerlab timing documents)
and consists of two parts (which are themselves calls): \call{dosado}
followed by \call{step to a wave}.\footnote{If you've read Bill
  Ackerman's ``Miscellaneous C4 Notions'' book
  (\url{http://lynette.org/sd/book3.pdf}), you might be objecting that
  \call{dosado} is fractionalizable, but not actually divisible into
  parts.  Dancers would likely balk at \call{initially tandem dosado
    to a wave}, for example.  Rest assured that the grammar can
  express this distinction; this example is slightly simplified.}
Indentation is significant, as are newlines.\footnote{Long lines can
  be split by using a backslash to escape the newline.}
Keywords (such as \clause{def}, \clause{in} and \clause{call}, usually
followed by a colon) are reserved only in the necessary contexts.
There is no problem defining or using the call
\call{pass in} for example, even though ``in'' is a keyword used both
to describe timing and in the syntax of the \clause{prim}  clause.

Here are two more simple definitions:
\begin{lstlisting}
def: couples circulate
  call: as couples(circulate)

def: couples trade
  in: 6
  call: as couples(trade)
\end{lstlisting}

You'll notice that the \clause{call} clauses contain what look like
function applications: the concept \call{as couples} is applied to
the call \call{circulate} or \call{trade}, respectively.  The
\call{couples trade} call explicitly specifies the number of beats of
the resulting call, but this is optional: \call{couples circulate}
uses the default timing.  In the case of the \call{as couples}
concept, the timing defaults to the timing of the base call, so
\call{couples circulate} takes 4 beats, just like \call{circulate}.

\subsection{Calls defined using primitives}
Let's look at another example to see how primitive dancer motions
ultimately make up a call definition.  Listing~\ref{lst:passthru}
shows the definition of the call \call{pass thru}.

\begin{lstlisting}[numbers=left,float,caption=A definition with primitives.,label=lst:passthru]
def: pass thru
  optional: LEFT
  in: 2
  // define the common case first: we've got two parts there
  from: FACING DANCERS
    prim: 0, 1, none
    prim: 0, 1, none
  // from mixed formation, this is indivisible
  from: mixed(FACING DANCERS, RH MINIWAVE)
    ipart:
      from: FACING DANCERS
        prim: 0, 1, none
        prim: 0, 1, none
      from: RH MINIWAVE
        call: step thru // "rear back and..."
\end{lstlisting}

The first line introduces the definition and names the call.
The \clause{optional} clause in the second line
contains a set of comma-separated tags which are used as hooks to trigger
the creation of additional call definitions or grammar rules.
In this case, the \texttt{LEFT} tag signifies that we will
automatically create an additional call \call{left pass thru} with
the definition \texttt{left(pass thru)}, where \call{left} is a
concept very similar (but not identical) to the C3B concept
\call{mirror}.\footnote{Collisions resolve to left-hand miniwaves when
  a call is done \call{mirror}.  They resolve to right-hand miniwaves
  when a call is done \call{left}.}
The third line says that this call will take 2 beats of music to execute.

The \clause{from} clauses in lines 5 and 9 introduce alternative
definitions for the call.  Each \clause{from} clause will be tried, in
sequence, until one succeeds.  The argument to clause{from} is
usually a formation name.  To be specific, the argument specifies a
\texttt{Matcher} expression, which takes a \texttt{Formation} and
returns a \texttt{FormationMatch} object.  For example, the
\texttt{MINIWAVE} matcher is equivalent to the matcher expression
\texttt{OR(RH\_MINIWAVE, LH\_MINIWAVE)}; that is, it matches either a
left-handed or a right-handed miniwave, setting dancer tags such as
\texttt{BEAU} and \texttt{BELLE} appropriately.  Available
\texttt{Matcher}s are defined in the
\package{net.cscott.sdr.calls.MatcherList} class in SDR.

\begin{figure}
\displaytwo
{ \dancer ~s \cr \dancer ~n }%
{\texttt{MatcherList.FACING\_DANCERS}}%
{ \dancer ~n & \dancer ~s }%
{\texttt{MatcherList.RH\_MINIWAVE}}
\caption{The patterns matched by the \textsf{FACING DANCERS} and
  \textsf{RH MINIWAVE} expressions.}
\label{fig:facingdancer}
\end{figure}

The first alternative, lines 5 through 7, applies when the formation
matches \texttt{FACING DANCERS}; that is, when every dancer in the
formation can be assigned to an instance of the \texttt{FACING
  DANCERS} pattern (see figure~\ref{fig:facingdancer}).  We specify
the motion as two separate primitives, to allow the call to be
fractionalized into halves.  The arguments to prim are, in order,
``sashay movement'' (positive is sashay to the right, negative is
sashay to the left), ``forward movement'' (positive is forward,
negative is backward), and ``turning'' (right, left, in, out,
none).\footnote{In order for turning direction to be unambiguous, the
  turn amount is never allowed to exceed a quarter turn in any one
  \clause{prim} clause.}  Dancers are nominally 2 steps apart in both
x and y axes after breathing.  Line 6 thus moves the dancer halfway
across by moving one step forward, and line 7 completes the
motion.\footnote{Note that ``do one half of a pass thru'' from facing
  dancers would result in a collision, which is resolved to right
  hands by the SDR engine as part of the breathing process.}

\begin{figure}
\displayone
{            &               & \dancer 2s \cr
  \ndancer 1n & \ngdancer 2s & \gdancer 1n }{}
\caption{A mixed collection of facing dancers and miniwaves.  The call
  \call{pass thru} is valid from this formation, but \call{do half a
    pass thru} is not.}
\label{fig:passthru}
\end{figure}
Line 9 introduces another alternative definition, which handles the
Callerlab ``Ocean Wave Rule''.  The \clause{from} expression is more
complex here: it uses the \texttt{mixed()} matcher combinator to match
a mixed collection of facing dancers and miniwaves.  Line 10 states that this
definition is not fractionalizable or divisible into parts, since it's
not clear what \call{do half of a pass thru} would mean for the
formation in figure~\ref{fig:passthru}.

Lines 11 through 13 deal with the \textsf{FACING DANCERS}
subformations in the mixed collection, and they give the same
definition as lines 5 through 7.   Line 14 says that from dancers in a
right-hand miniwave, pass thru is the same as step thru.

\subsection{Calls with arguments}

Before we deal with the definition language more rigorously, let's walk through
a more complicated example.  Listing~\ref{lst:squarethru} shows
the definition of the call \call{square thru}.\footnote{Again, the listing is
  slightly simplified, since the actual definition adds a bit of
  complexity to ensure that \call{square thru $N$} has exactly
  $N$ parts.}

\begin{lstlisting}[numbers=left,float,caption=A definition with arguments.,label=lst:squarethru]
def: square thru(n=4)
  optional: LEFT
  spoken: square thru (<n=number> (hands (around|round)?)?)?
  from: ANY
    condition: and(greater([n],0),not(greater([n],1)))
    call: _fractional([n], _in(2, pull by))
  from: ANY
    condition: greater([n], 1)
    call: _in(2, _square thru part)
    call: left(square thru(_subtract num([n],1)))
\end{lstlisting}

This is a recursive definition.  Let's take it line by line.

The \clause{def} clause in line 1 says that this is a definition of a call named
``square thru'' which takes a single parameter, which we will refer to
as $n$ in the body of the definition.  The parameter has a
default value, 4.  If no argument is supplied, $n$ will be set
to 4 when the definition is evaluated.

The \clause{optional} clause in the second line specifies that we will
automatically create an
additional call \call{left square thru(n)} with the definition
\call{left(square thru([n]))}, as described in the previous example.

The \clause{spoken} clause in line 3 provides a regular-expression
style grammar for a set of spoken words which are considered to refer
to this definition.  The \clause{spoken} clause is optional for calls
with no arguments, defaulting to the call name.\footnote{Calls whose
  names begin with an underscore are considered ``internal''; they do
  not have a default spoken representation.  Nothing prevents you from
  adding a \clause{spoken} clause for a call whose name begins with an
  underscore, but this is not encouraged.}  The \clause{spoken} clause
also names the grammar productions used for arguments.  According to
line 3, speaking ``square thru three'' (or writing ``square thru 3'')
will result in the function application \texttt{square
  thru(3)}.\footnote{``Function names'' can contain spaces, as this
  one does.  In general, identifiers consist of a \textit{sequence} of
  ``words'' separated by whitespace.}  The clause allows ``square
thru'' to be an acceptable spoken version of the call; the numeric
argument takes its default value of 4 since it is not matched in the
grammar.  Other valid written/spoken word sequences are ``square thru 2 1/2''
and ``square thru three hands round''.

The \clause{from} clauses in lines 4 and 7 introduce alternatives, as
described in the previous section.  In this case, the \texttt{ANY}
matcher in line 4 matches any formation.  It is used here to introduce
one of two alternative definitions for the call without otherwise
imposing any constraints.  Usually alternative definitions are
distinguished by the formations from which they are applied, but in
the case of ``square thru'' we are going to use a \clause{condition}
clause to distinguish the alternatives.

The \clause{condition} clause in line 5 evaluates a boolean
expression, halting evaluation of this alternative\footnote{Evaluation
  is halted by throwing a \texttt{BadCallException}.} if the condition
evaluates to false.  Available boolean functions are defined in
\package{net.cscott.sdr.calls.PredicateList}.  Note that arguments to
boolean predicates can establish other, non-boolean, contexts.  In
this case the \call{greater} predicate establishes a numeric context
for the evaluation of \texttt{0}, \texttt{1}, and \texttt{[n]} (which
is a reference to the argument named $n$).  A predicate expression,
like \texttt{formation(MINIWAVE)}, could also establish
\texttt{Matcher} or other contexts for the evaluation of arguments.
The particular expression in line 5 restricts this alternative to
apply only in the base case of the definition: ``square thru $n$'' for
$0 < n \leq 1$.

The \clause{call} clause in line 6 provides the definition
applicable for this alternative: do the fraction $n$ of a
``pull by'' taking 2 beats.  That is, if $n=1/2$, then do half of a
pull by and take 1 beat.  (``Pull by'' is the same as a ``right pull by.'')

Lines 7 through 10 introduce a new alternative, applicable when $n>1$
(as specified in line 8).
It begins with the internal call \call{\_square thru part}, done in 2
beats (line 9).  This will be given its definition (``pass thru and face in'')
elsewhere in the file.  After this part of the call is complete, on
line 10 we recursively
invoke the call ``left square thru $(n-1)$'', written as an application
of the \call{left} concept to a recursive invocation of \call{square thru}.

\section{Idioms}
There are a number of nonstandard internal concepts which are used
frequently inside SDR definitions.

\subsection{\_quasi concentric}
The \call{\_quasi concentric} concept is like \call{concentric},
except that no ``lines to lines, columns to columns'' or ``opposite
elongation rule'' adjustment is done.  The elongation of the resulting
formation is exactly the same as the original formation.  This is
often used in calls which are informally defined with ``centers'' and
``ends''.   For example, from facing lines ``centers face left while the
ends face left'' ends in a right-hand column, while ``concentric face
left'' ends in a two-faced line due to the lines-to-lines rule.  These
two motions are expressed in SDR as \call{\_quasi concentric(face left,
  face left)} and \call{concentric(face left, face left)},
respectively.

\displayone
{ \gdancer 1s & \dancer 1s & \gdancer 2s & \dancer 2s \cr
  \dancer 4n & \gdancer 4n & \dancer 3n & \gdancer 3n }%
{Starting formation}
\displaytwo
{ \gdancer 1e & \dancer 1e & \gdancer 2e & \dancer 2e \cr
  \dancer 4w & \gdancer 4w & \dancer 3w & \gdancer 3w }%
{After ``centers face left while ends face left''}
{ \gdancer 1e & \dancer 2e \cr
  \dancer 1e & \gdancer 2e \cr
  \gdancer 4w & \dancer 3w \cr
  \dancer 4w & \gdancer 3w }%
{After ``concentric face left''}

The \call{\_quasi concentric} concept is also used to define
concentric variants such as \call{concentric triple boxes} which
work to spots.

\subsection{\_o concentric}
The \call{\_o concentric} is similar to \call{\_quasi concentric}, in
that it does no elongation adjustment.  It goes further, however, in
that the ends are not expected to mentally ``breathe together'' before
finding the formation for their call (as the ends would do to find
their box when executing ``concentric scoot back'' from two-faced
lines, for example).  When an informal definition says ``centers cross
back while the ends O circulate twice''\footnote{This is ``like a
  settle back'', a C4 call.} we can't use the \call{concentric} or
\call{\_quasi concentric} concepts, because the ends won't have the
required O formation after they've mentally breathed in.  Although the
ends are not working concentrically, the centers part is still
concentric: ``own the centers cross back by twice O circulate'' is not
what is intended.  ``Own the centers concentric cross back by twice O
circulate'' is a correct precisely-stated definition, but evaluating
the call in that manner creates complications: when evaluating the
\call{own} concept we need to create new formations with phantom
centers/ends to allow everyone to successfully execute a O
circulate/concentric cross back.  We use the simpler definition
\call{\_o concentric(cross back, 2(o circulate))} instead.

\displayone
{ \dancer 1n & \gdancer 2s & \gdancer 1n & \dancer 2s \cr
  \dancer 4n & \gdancer 3s & \gdancer 4n & \dancer 3s }%
{Starting formation}
\displaytwo
{ \dancer 1n & \dancer 2s \cr
  \dancer 4n & \dancer 3s }%
{\ctablebox{Ends evaluate in this formation \\ for \call{concentric} or
  \call{\_quasi concentric}.}}
{ \dancer 1n & \pdancer ~x & \pdancer ~x & \dancer 2s \cr
  \dancer 4n & \pdancer ~x & \pdancer ~x & \dancer 3s }%
{\ctablebox{Ends evaluate in this formation \\ for
  \call{\_o concentric}.}}

It may be that the ends part of \call{zip code} is also easier to define using
\call{\_o concentric}, although you might also imagine defining the
ends part as a series of ``pass thru, concentric face in'', using the
lines-to-lines rule to good effect.

\subsection{\_maybe touch}\label{sec:maybetouch}
The \call{\_maybe touch} pseudo-concept is used to wrap the first part
of a call where the Callerlab ``Facing Couples Rule'' applies.  This
ensures that ``swing thru'' is valid from facing couples and from
mixed formations:
\displayone
{            &            & \dancer 2s &\gdancer 3s \cr
  \dancer 1n &\gdancer 2s &            &            &\gdancer 4n & \dancer 3s\cr
             &            &\gdancer 1n & \dancer 4n}%
{
Legal starting formation for \call{grand swing thru}.
}


Recall that one distinction between the call \call{swing and mix} and the
two calls \call{swing} and \call{mix} is that the facing couples rule
applies to \call{swing and mix} but it does not apply to the A2 call
\call{swing}, which must be called from a wave.  SDR uses the following
definition for \call{swing and mix} to express this distinction:
\begin{lstlisting}
def: swing and mix // 3 part call
  part:
    call: _in(2, _maybe touch(_wave swing))
  part: 2
    call: mix
\end{lstlisting}

In this definition, the call \call{\_wave swing} is A2 \call{swing},
valid only from general lines where the ends and their adjacent
centers are facing opposite directions.  The call \call{swing} is
defined to do either a mainstream partner swing or \call{\_wave swing},
depending on context.

The \call{\_maybe touch} helper is also used to implement Callerlab's
similar ``Ocean Wave Rule'' when defining \call{box the gnat}.

\subsection{\_designator concept}
It is worth reading Dan Neumann's article \textit{Designators and Concepts}
(\url{http://lynette.org/dconcepts.html}) for context.
Designators are terms such as ``heads'', ``boys'', ``centers'', and
``leaders''.\footnote{SDR distinguishes between ``primitive'' or
  ``intrinsic'' tags which are immutable properties of the dancer
  (``head'', ``girl'') and tags which are properties of the formation
  (``center'', ``belle'').}
Some calls require a mandatory designator (\call{boys run},
\call{beaus advance to a column}) and some calls will accept an
optional designator (\call{walk and dodge}, \call{trade}).  Other
calls used with a designator can be interpreted as an application of
a ``designator concept'', with somewhat idiosyncratic semantics.

\begin{figure}
\displaythree
{ \dancer 1e \cr \gdancer 2w \cr \gdancer 1e \cr \dancer 2w \cr
  \dancer 4e \cr \gdancer 3w \cr \gdancer 4e \cr \dancer 3w }
{\ctablebox{Starting \\ formation}}
{ & \dancer 1e \cr & \gdancer 2w \cr
 \gdancer 3e & & \dancer 2e \cr
 \dancer 4w & & \gdancer 1w \cr
  & \gdancer 4e \cr & \dancer 3w }
{\call{Centers ah so}}
{ \gdancer 2n & \dancer 1s \cr
  \dancer 2s & \gdancer 3n \cr
  \gdancer 1s & \dancer 4n \cr
  \dancer 3n & \gdancer 4s }
{\call{Initially centers, reset 1/2}}
\caption{Two-call sequence using the designator \call{centers} as a concept.}
\label{fig:designator}
\end{figure}

\begin{figure}
\displaythree
{          & \dancer 1n & \gdancer 1n & \cr
\gdancer 2n & \dancer 4s & \dancer 2n & \gdancer 4s \cr
           & \gdancer 3s & \dancer 3s }%
{Before}
{          & \gdancer 1s & \dancer 1s & \cr
\dancer 4n & \gdancer 2s & \gdancer 4n & \dancer 2s \cr
           & \dancer 3n & \gdancer 3n }%
{Halfway}
{          & \gdancer 1s & \dancer 1s & \cr
\dancer 2n & \gdancer 2s & \gdancer 4n & \dancer 4s \cr
           & \dancer 3n & \gdancer 3n }%
{After}
\caption{\call{finally boys half crazy trade}}
\label{fig:finallyboys}
\end{figure}

Figure~\ref{fig:designator} shows two calls which use the designator
\call{centers} as a concept.  The first call demonstrates that the
meaning of the \call{centers} concept is that the designated center
dancers are to do their part of the call concentrically, remaining in
the center of the formation.  That is, \call{centers ah so} is
different from \call{own the centers, ah so by nothing}.  The use of
\call{initially centers} in the second call shows that \call{centers}
is a first-class concept; it can be curried to be applied as part of
metaconcepts like other concepts.

SDR defines a three-argument concept \call{\_designator concept(anyone,
  anything1, anything2=nothing)}.  It is usually spoken as
``\textit{anyone} \textit{anything1} (while the others \textit{anything2})''.
We define it formally as follows.

First, we check if \textit{anything1} is one of a small number of ``special''
calls which take optional designators.  ``Boys trade'' often has a
special ``trade down the line'' meaning, which is not otherwise
captured by the ``concentric'' or ``do your part'' alternatives for
the designator concept.  If the \textit{anything1} call is one of these
special cases and \textit{anything2} is \call{nothing}, we attempt to
invoke the \textit{anything1} call with its optional
designator.  This allows \call{finally boys half crazy trade} from a
three-quarter tag with a GBBG wave in the center (figure~\ref{fig:finallyboys}).

Then we check to see if the designated dancers are the center 2,
center 4, or center 6 dancers.  If they are, we do the call using the
appropriate variant of the \call{\_quasi concentric} concept.  For
symmetry, we also treat the case where the designated dancers are the
outer 2, 4, or 6, so that \call{center six trade while the
  others u-turn back} and \call{outside two u-turn back while the
  others trade} result in the same interpretation.  Note that we are
looking at the designated dancers, not just the designator: \call{head
  boys shazam}, \call{very centers shazam} and \call{center two
  shazam} are handled identically if the head boys are the very centers.

We check to see if the designator is ``everyone'' (in which case we do
the \textit{anything1} call ignoring the concept) or ``no one'' (in
which case we do the \textit{anything2} call).
This is mostly to handle caller patter such as \call{everyone
up to the middle and back}, but it's a reasonable generalization to
allow \call{initially nobody swing thru}, for example.

\begin{figure}
\displaythree
{ \dancer 1n & \dancer 2s & \gdancer 2n & \gdancer 1s \cr
  \gdancer 3n & \gdancer 4s & \dancer 4n & \dancer 3s }%
{Before}
{ \dancer 2n & \dancer 1s & \gdancer 2n & \gdancer 1s \cr
  \gdancer 3n & \gdancer 4s & \dancer 3n & \dancer 4s }%
{Halfway}
{ \dancer 2n & \gdancer 2s & \dancer 1n & \gdancer 1s \cr
  \gdancer 3n & \dancer 3s & \gdancer 4n & \dancer 4s }%
{After}
\caption{\call{initially boys swing thru}}
\label{fig:initiallyboys}
\end{figure}

If none of the above steps yielded a successful application, we
evaluate the concept as ``own the \textit{anyone},
  \textit{anything1} by \textit{anything2}''; that is, as \call{own
    the([anyone], [anything1], [anything2])}.  This final case catches
  usage such as \call{initially boys swing thru} from BBGG right-hand
  ocean waves (figure~\ref{fig:initiallyboys}).

Note that formation-specific designators such as ``leads'' and
``beaus'' are somewhat problematic, as these designators are not
applied until after a formation match is done.  It may eventually be
necessary to either look inside the \textit{anything} call for
appropriate formation matchers, or else to apply general matchers for
beau/belle/lead/trailer.  This would probably be done inside the
\call{own the} concept, as it wouldn't be appropriate for the
\textit{anything} call to be done concentrically just because all the
designated dancers happened to be in the center.  Or would it? From parallel
right-hand two-faced lines, is
\call{belles scoot back} reasonable (figure~\ref{fig:bellesscoot})?

\begin{figure}
\displayone
{ \dancer 1n & \gdancer 1n & \gdancer 2s & \dancer 2s \cr
  \dancer 4n & \gdancer 4n & \gdancer 3s & \dancer 3s }%
{}
\caption{Is \call{belles scoot back} reasonable from this formation?}
\label{fig:bellesscoot}
\end{figure}

We also currently special-case the designators ``heads'' and ``sides''
from a squared set. From a squared set \call{heads square thru} means
``heads go into the middle and square thru'', or more precisely,
\call{heads (press ahead ; square thru)}.  Adding this case to the
\call{\_designator concept} allows \call{echo heads square thru}
from a squared set.

\section{Clause Reference}

\subsection{Definition and grammar clauses}
\subsubsection{def}
\subsubsection{optional}
\subsubsection{spoken}
\subsubsection{example}
% main part of definition is ``piece'' which is restriction, opt, par, seq

\subsection{Selector and alternative clauses}
\subsubsection{from}
\subsubsection{select}

\subsection{Timing and restriction clauses}
\subsubsection{in}
\subsubsection{condition, ends in}

\subsection{Sequence clauses}
\subsubsection{part, ipart, xpart, fpart}
\subsubsection{call}
\subsubsection{prim}

\end{document}
